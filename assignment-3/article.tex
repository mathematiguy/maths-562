%%%%%%%%%%%%%%%%%%%%%%%%%%%%%%%%%%%%%%%%%
% Arsclassica Article
% LaTeX Template
% Version 1.1 (1/8/17)
%
% This template has been downloaded from:
% http://www.LaTeXTemplates.com
%
% Original author:
% Lorenzo Pantieri (http://www.lorenzopantieri.net) with extensive modifications by:
% Vel (vel@latextemplates.com)
%
% License:
% CC BY-NC-SA 3.0 (http://creativecommons.org/licenses/by-nc-sa/3.0/)
%
%%%%%%%%%%%%%%%%%%%%%%%%%%%%%%%%%%%%%%%%%

%----------------------------------------------------------------------------------------
%	PACKAGES AND OTHER DOCUMENT CONFIGURATIONS
%----------------------------------------------------------------------------------------

\documentclass[
10pt, % Main document font size
a4paper, % Paper type, use 'letterpaper' for US Letter paper
oneside, % One page layout (no page indentation)
%twoside, % Two page layout (page indentation for binding and different headers)
headinclude,footinclude, % Extra spacing for the header and footer
BCOR5mm, % Binding correction
]{scrartcl}
\renewcommand{\labelenumii}{\theenumii}
\renewcommand{\theenumii}{\theenumi.\arabic{enumii}.}
\usepackage{mdframed}
\usepackage{amsmath}
\DeclareMathOperator*{\argmax}{argmax} % thin space, limits underneath in displays
\DeclareMathOperator*{\argmin}{argmin} % thin space, limits underneath in displays
\DeclareMathOperator*{\sig}{sig}

%%%%%%%%%%%%%%%%%%%%%%%%%%%%%%%%%%%%%%%%%
% Arsclassica Article
% Structure Specification File
%
% This file has been downloaded from:
% http://www.LaTeXTemplates.com
%
% Original author:
% Lorenzo Pantieri (http://www.lorenzopantieri.net) with extensive modifications by:
% Vel (vel@latextemplates.com)
%
% License:
% CC BY-NC-SA 3.0 (http://creativecommons.org/licenses/by-nc-sa/3.0/)
%
%%%%%%%%%%%%%%%%%%%%%%%%%%%%%%%%%%%%%%%%%

%----------------------------------------------------------------------------------------
%	REQUIRED PACKAGES
%----------------------------------------------------------------------------------------

\usepackage[
nochapters, % Turn off chapters since this is an article        
beramono, % Use the Bera Mono font for monospaced text (\texttt)
eulermath,% Use the Euler font for mathematics
pdfspacing, % Makes use of pdftex’ letter spacing capabilities via the microtype package
dottedtoc % Dotted lines leading to the page numbers in the table of contents
]{classicthesis} % The layout is based on the Classic Thesis style

\usepackage{arsclassica} % Modifies the Classic Thesis package

\usepackage[T1]{fontenc} % Use 8-bit encoding that has 256 glyphs

\usepackage[utf8]{inputenc} % Required for including letters with accents

\usepackage{graphicx} % Required for including images
\graphicspath{{Figures/}} % Set the default folder for images

\usepackage{enumitem} % Required for manipulating the whitespace between and within lists

\usepackage{lipsum} % Used for inserting dummy 'Lorem ipsum' text into the template

\usepackage{subfig} % Required for creating figures with multiple parts (subfigures)

\usepackage{amsmath,amssymb,amsthm} % For including math equations, theorems, symbols, etc

\usepackage[round]{natbib}   % omit 'round' option if you prefer square brackets

\usepackage{varioref} % More descriptive referencing

\bibliographystyle{plainnat}

%----------------------------------------------------------------------------------------
%	THEOREM STYLES
%---------------------------------------------------------------------------------------

\theoremstyle{definition} % Define theorem styles here based on the definition style (used for definitions and examples)
\newtheorem{definition}{Definition}

\theoremstyle{plain} % Define theorem styles here based on the plain style (used for theorems, lemmas, propositions)
\newtheorem{theorem}{Theorem}

\theoremstyle{remark} % Define theorem styles here based on the remark style (used for remarks and notes)

%----------------------------------------------------------------------------------------
%	HYPERLINKS
%---------------------------------------------------------------------------------------

\hypersetup{
%draft, % Uncomment to remove all links (useful for printing in black and white)
colorlinks=true, breaklinks=true, bookmarks=true,bookmarksnumbered,
urlcolor=webbrown, linkcolor=RoyalBlue, citecolor=webgreen, % Link colors
pdftitle={}, % PDF title
pdfauthor={\textcopyright}, % PDF Author
pdfsubject={}, % PDF Subject
pdfkeywords={}, % PDF Keywords
pdfcreator={pdfLaTeX}, % PDF Creator
pdfproducer={LaTeX with hyperref and ClassicThesis} % PDF producer
}
 % Include the structure.tex file which specified the document structure and layout

\hyphenation{Fortran hy-phen-ation} % Specify custom hyphenation points in words with dashes where you would like hyphenation to occur, or alternatively, don't put any dashes in a word to stop hyphenation altogether

%----------------------------------------------------------------------------------------
%	TITLE AND AUTHOR(S)
%----------------------------------------------------------------------------------------

%\subtitle{Subtitle} % Uncomment to display a subtitle

\author{\spacedlowsmallcaps{Caleb Moses*}} % The article author(s) - author affiliations need to be specified in the AUTHOR AFFILIATIONS block

\date{} % An optional date to appear under the author(s)


\newenvironment{problem}[2][]
               { \begin{mdframed}[backgroundcolor=gray!20] \textbf{#1 #2} \\}
               {  \end{mdframed}}

% Define solution environment
\newenvironment{solution}
               {\textit{Solution:}}
               {}

%----------------------------------------------------------------------------------------

\title{MATH/COMP 562 ASSIGNMENT 3}
\author{Caleb Moses}

\begin{document}

\maketitle

\section*{Rademacher Complexity}
\subsection*{Exercise 3.1 (Rademacher identities)}
\begin{problem}{Mohri Exercise 3.8. part (c) only}
Fix $m \geq 1$. Prove the following identity for any $\alpha \in \mathbb{R}$ and any two hypothesis sets $\mathcal{H}$ and $\mathcal{H}_0$ of functions mapping from $\mathcal{X}$ to $\mathbb{R}$:

$\mathfrak{R}_m(\{\max(h, h_0 ): h \in \mathcal{H}, h_0 \in \mathcal{H}_0 \}) \leq \mathfrak{R}_m (\mathcal{H}) + \mathfrak{R}_m (\mathcal{H}_0)$, where $\max(h, h_0 )$ denotes the function $x \mapsto max_{x\in \mathcal{X}} (h(x), h_0 (x))$ (Hint: you could use the identity $\max(a, b) = \frac{1}{2} [a + b + |a - b|]$ valid for all $a, b \in \mathbb{R}$ and Talagrand’s contraction lemma (see lemma 5.7)).
\end{problem}

\begin{solution}
Let $\mathcal{H}_{\max} = \{\max(h, h_0 ): h \in \mathcal{H}, h_0 \in \mathcal{H_0} \}$ be the class of functions that take the maximum of the predictions of a function $h$ from $\mathcal{H}$ and a function $h_0$ from $\mathcal{H}_0$.

Now we calculate the Rademacher complexity of $\mathcal{H}_{\max}$:

\begin{align}
\mathfrak{R}_m(\mathcal{H}_{\max}) &= \mathbb{E} \left[ \sup_{h\in\mathcal{H}, h_0\in\mathcal{H}_0} \frac{1}{m} \sum_{i=1}^m \sigma_i \max(h(z_i), h_0(z_i)) \right] \\
&= \mathbb{E} \left[ \sup_{h\in\mathcal{H}, h_0\in\mathcal{H}_0} \frac{1}{m} \sum_{i=1}^m \frac{\sigma_i}{2} (h(z_i) + h_0(z_i) + |h(z_i) - h_0(z_i)|) \right] \\
&\leq \frac{1}{2} \mathbb{E}\left[ \sup_{h\in\mathcal{H}}\frac{1}{m} \sum_{i=1}^m \sigma_i h(z_i)\right] + \mathbb{E}\left[\sup_{h_0\in\mathcal{H}_0}\frac{1}{m} \sum_{i=1}^m \sigma_i h_0(z_i)\right] + \\
&\qquad \frac{1}{2} \mathbb{E}\left[\sup_{h\in\mathcal{H}, h_0\in\mathcal{H}_0} \frac{1}{m} \sum_{i=1}^m \sigma_i |h(z_i) - h_0(z_i)|\right] \\
&\leq \frac{1}{2} \mathbb{E}\left[ \sup_{h\in\mathcal{H}}\frac{1}{m} \sum_{i=1}^m \sigma_i h(z_i)\right] + \frac{1}{2}\mathbb{E}\left[\sup_{h_0\in\mathcal{H}_0}\frac{1}{m} \sum_{i=1}^m \sigma_i h_0(z_i)\right] + \\
&\qquad \frac{1}{2} \mathbb{E}\left[\sup_{h\in\mathcal{H}} \frac{1}{m} \sum_{i=1}^m \sigma_i |h(z_i)|\right] + \frac{1}{2} \mathbb{E}\left[\sup_{h_0\in\mathcal{H}_0} \frac{1}{m} \sum_{i=1}^m \sigma_i |h_0(z_i)|\right] \\
&\leq \frac{\mathfrak{R}_m(\mathcal{H})}{2} + \frac{\mathfrak{R}_m(\mathcal{H}_0)}{2} + \frac{1 \cdot \mathfrak{R}_m(\mathcal{H})}{2} + \frac{1 \cdot \mathfrak{R}_m(\mathcal{H}_0)}{2} \\
&= \mathfrak{R}_m(\mathcal{H}) + \mathfrak{R}_m(\mathcal{H}_0)
\end{align}

(1) is the definition of Rademacher complexity for $\mathcal{H}_{\max}$. We take advantage of the hint identity in (2) and then in (3) and (4) we apply the linearity of expectation and sub-additivity of the supremum. In (6) we apply the triangle inequality to the absolute difference between $h$ and $h_0$, and use sub-additivity of the supremum and linearity of expectation again to split it into two expectations. Then in (7) we use the definition of Rademacher complexity to rewrite the first two terms, and the last two require Talagrand's contraction lemma with the fact that the Lipschitz constant of the absolute value function is 1. Simplifying gives the sum of Rademacher complexities of $\mathcal{H}$ and $\mathcal{H}_0$ as desired.

\end{solution}

\section{VC dimension}

\subsection*{Exercise 3.2 (VC dimension of finite hypothesis sets)}
\begin{problem}{Mohri Exercise 3.14}
VC-dimension of finite hypothesis sets. Show that the VC-dimension of a finite hypothesis set $\mathcal{H}$ is at most $\log_2|\mathcal{H}|$.
\end{problem}

\begin{solution}
  Let $\mathcal{H}$ be a finite hypothesis set that shatters an instance space $X$. Then each element of $\mathcal{H}$ corresponds to a possible labelling of $X$ and there are $2^{|X|}$ such labellings. Since $\mathcal{H}$ shatters $X$ we require $|\mathcal{H}| \geq 2^{|X|}$, since we need at least as many hypotheses as there are labellings for $X$. Equivalently, $\log_2{|\mathcal{H}|} \geq |X|$ so the largest set $X$ that $\mathcal{H}$ can shatter has size $\log_2 |\mathcal{H}|$ which gives the desired upper bound on the VC-dimension.
\end{solution}

\subsection*{Exercise 3.3 (VC dimension of closed balls).}
\begin{problem}{Mohri Exercise 3.17}
Show that the VC-dimension of the set of all closed balls in $\mathbb{R}_n$, i.e. sets of the form ${x \in \mathbb{R}_n : ||x-x_0|| \leq r}$ for some $x_0 \in \mathbb{R}_n$ and $r \geq 0$, is less than or equal to $n + 2$.
\end{problem}

\begin{solution}
Fix $r>0$, then the constraint $||x-a||^2<r$ is equivalent to:
\begin{align*}
  ||x-a||^2 - r &< 0 \\
  (x-a)^T(x-a) - r &< 0 \\
  ||x||^2 - 2(x\cdot a) + ||a||^2 - r &< 0 \\
  \sum_{i=1}^n x_i^2 - 2 \sum_{i=1}^n a_i x_i + \sum_{i=1}^n a_i^2 - r &< 0
\end{align*}
\end{solution}

We will show that there are enough free variables to capture $n+1$ arbitrary points. The last line is equivalent to $WX - B < 0$ with:

\begin{align*}
  W &= \begin{bmatrix}
    1 \\
    -2a_1 \\
    ... \\
    -2a_n
   \end{bmatrix} & X = \begin{bmatrix}
     \sum_{i=1}^n x_i^2 \\
     x_1 \\
     ... \\
     x_n
   \end{bmatrix} &&  B = \sum_{i=1}^n r - a_i^2
\end{align*}

There are n+1 linearly independent equations with n+1 unknowns (1 for each $a_i$ from 1 to n, plus $r$). So the VC dimension of closed balls in $\mathbb{R}^n$ is at most equal to the VC dimension of hyperplanes in $\mathbb{R}^{n+1}$, which is $n+2$.
\newpage
\subsection*{Exercise 3.4 (Function class with infinite VC dimension).}
\begin{problem}{Mohri Exercise 3.20}
Consider the hypothesis family of sine functions: $\mathcal{H} = \{x \to sin(\omega x): \omega \in \mathbb{R}\}$.
\begin{enumerate}[label=(\alph*)]
\item Show that for any $x \in \mathbb{R}$ the points $x$, $2x$, $3x$ and $4x$ cannot be shattered by this family of sine functions.
\item Show that the VC-dimension of the family of sine functions is infinite. (Hint: show that $\{2^{-i}: i \leq m\}$ can be shattered for any $m > 0$.)
\end{enumerate}
\end{problem}

\begin{solution}
\begin{enumerate}[label=(\alph*)]
  \item Fix $x\in\mathbb{R}$. Consider the dichotomy $\{(x, -), (2x, -), (3x, +), (4x, -)\}$. Assume towards contradiction that $\mathcal{H}$ realises this dichotomy, so there exists $\omega\in\mathbb{R}$ satisfying $\sin(\omega x) < 0$, $\sin(2 \omega x) < 0$, $\sin(3 \omega x) > 0$ and $\sin(4 \omega x) < 0$.

  Using the double angle formulae for sin and cos gives:
  \begin{align*}
    \sin(4\omega x) &= 2 \sin(2\omega x)\cos(2\omega x) \\
    &= 2\sin(2\omega x) (1 - 2\sin^2(\omega x))
  \end{align*}

  Since $\sin(4\omega x) < 0$ and $\sin(2\omega x) < 0$ by assumption, $1 - 2\sin^2(\omega x) > 0$ must hold otherwise the right hand side is positive while the left hand side is negative. Therefore $\sin^2(\omega x) < \frac{1}{2}$.

  Next we use the triple angle formula for $\sin$ to calculate the following:
  \begin{align*}
    \sin(3\omega x) &= \sin(\omega x)(3 - 4\sin^2(\omega x))
  \end{align*}

  Since $\sin(3\omega x) > 0$ and $\sin(\omega x) < 0$, we require $3 - 4\sin^2(\omega x) < 0$ otherwise the left hand side is positive and the right hand side is negative. Therefore $\sin^2(\omega x) > \frac{3}{4}$. This contradicts $\sin^2(\omega x) < \frac{1}{2}$, so there is no $\omega$ such that $\mathcal{H}$ shatters $x$, $2x$, $3x$ and $4x$.

\item Let $m \in \mathbb{N}$ and consider the points $\{2^{-i}: i \leq m\}$ with arbitrary labels $(y_1, ... y_{m}) \in \{-1, +1\}^m$. Let $y_i^\prime = \frac{1-y_i}{2}$ and choose $\omega = \pi(1 + \sum_{i=1}^m 2^{i}y_i^\prime)$. We will show that this is sufficient to correctly classify our chosen labels. then for any $j\in[1,m]$ we have the following:
    \begin{align*}
      \omega 2^{-j} &= 2^{-j} \pi\left[1 + \sum_{i=1}^m 2^{i}y_i^\prime\right] \\
      &= \pi \left[2^{-j} + \sum_{i=1}^m 2^{i-j}y_i^\prime\right] \\
      &= \pi \left[2^{-j} + y_j^\prime + \sum_{i=1}^{j-1} 2^{i-j}y_i^\prime + \sum_{i=j+1}^m 2^{i-j}y_i^\prime \right] \\
      &= \pi \left[2^{-j} + y_j^\prime + \sum_{i=1}^{j-1} 2^{-i}y_{j-i}^\prime + \sum_{i=1}^{m-j} 2^iy_{i+j}^\prime \right] \\
      &= \pi \left[2^{-j} + y_j^\prime + \sum_{i=1}^{j-1} 2^{-i}y_{j-i}^\prime\right] + 2\pi \sum_{i=1}^{m-j} 2^{i-1}y_{i+j}^\prime
    \end{align*}
  The last term contributes multiples of $2\pi$ to $\omega 2^{-i}$ and so can be ignored. Since $y_i\in\{0,1\}$, the remaining term can be bound in the following way:

  \begin{align*}
    \pi \left[2^{-j} + y_j^\prime + \sum_{i=1}^{j-1} 2^{-i}y_{j-i}^\prime\right] &\leq \pi \left[ y_j^\prime + \sum_{i=1}^{j} 2^{-i}\right] < \pi(1 + y_j^\prime) \\
    \pi \left[2^{-j} + y_j^\prime + \sum_{i=1}^{j-1} 2^{-i}y_{j-i}^\prime\right] &> \pi y_j^\prime
  \end{align*}

  Now if $y_j = 1$ then $y_j^\prime = 0$ and $0 < \omega 2^{-j} < \pi$ which implies $\sin(\omega 2^{-j}) > 0$. Similarly, if $y_j = -1$, then $y_j^\prime = 1$, $\pi < \omega 2^{-j} < 2\pi$ and we have that $\sin(\omega 2^{-j}) < 0$ as desired.

\end{enumerate}
\end{solution}

\section*{Stability Theory}
Recall that a function $f$ is $\lambda$-strongly convex if for all $u, w \in \mathbb{R}^d , \alpha \in [0, 1]$:
$$f(\alpha w + (1 - \alpha)u) + \frac{\lambda}{2} \alpha(1 - \alpha)\|w - u\|^2 \leq \alpha f(w) + (1 - \alpha)f(u)$$
\begin{problem}{Exercise 3.5 (Strong convexity)}
  \begin{enumerate}[label=(\alph*)]
    \item Prove that $f(w) = \|w\|^2$ is $\lambda$-strongly convex with $\lambda = 2$.
    \item Give an example of a convex function which is not $\lambda$-strongly convex for any $\lambda > 0$.
    \item Let $f : \mathbb{R}^d \rightarrow \mathbb{R}$ be convex. Prove that $g(w, \lambda) = f(w) + \frac{\lambda}{2}\|w\|^2$ is $\lambda$-strongly convex.
    \item Prove that $g$ is $\lambda$-strongly convex if and only if $f(w) = g(w) - \frac{\lambda}{2}\|w\|^2$ is convex.
  \end{enumerate}
\end{problem}

\begin{solution}
  \begin{enumerate}[label=(\alph*)]
      \item We begin by calculating the left hand side of the strong convexity inequality:
        \begin{align*}
        &\hphantom{=} ||\alpha w + (1-\alpha)u)||^2 + \alpha(1-\alpha)||w-u||^2 \\
        &= (\alpha w + (1-\alpha)u)^T(\alpha w + (1-\alpha)u) + \alpha(1-\alpha)(w-u)^T(w-u) \\
        &= \alpha^2||w||^2 + 2\alpha(1-\alpha)(u\cdot w) + (1-\alpha)^2||u||^2 + \alpha(1-\alpha)(||w||^2 - 2(u\cdot w) + ||u||^2) \\
        &= (\alpha^2+\alpha(1-\alpha))||w||^2 + ((1-\alpha)^2 + \alpha(1-\alpha))||u||^2 \\
        &= \alpha||w||^2 + (1-\alpha)||u||^2
        \end{align*}
      \item I claim that $f(x) = x^4$ is convex, but not $\lambda$-strongly convex for any $\lambda>0$. We know $f$ is convex since $f^{\prime\prime}(x) = 12x^2 > 0$ for all x.
        Now to show $f$ is not $\lambda$-strongly convex, choose $x_0=0$, $x_1=w$ and $\alpha=\frac{1}{2}$. Suppose towards contradiction that $f$ is $\lambda$-strongly convex for some $\lambda > 0$. Then the following must hold for all $w\in\mathbb{R}$:
        \begin{align*}
          \left(\frac{w}{2}\right)^4 + \frac{\lambda}{8}w^2 &\leq \frac{1}{2}w^4 \\
          \lambda &\leq \frac{8}{w^2}\left(\frac{w^4}{2} - \frac{w^4}{16}\right) = 3.5w^2
        \end{align*}
        $\lambda$ is fixed but $w$ was chosen freely so by choosing $w$ arbitrarily close to zero we can obtain a contradiction. To this end, choose $w = \sqrt{\lambda/7}$. Then we have:
        $$\lambda \leq 3.5\sqrt{\frac{\lambda}{7}}^2 \implies \lambda \leq \frac{\lambda}{2} $$
        This concludes the proof.
      \item Let $g = f(w) + \frac{\lambda}{2}||w||^2$, then we can calculate the following: \begin{align*}
        &\hphantom{=} g(\alpha w + (1-\alpha)u, \lambda) + \frac{\lambda}{2}\alpha(1-\alpha)||w-u||^2 \\
        &= f(\alpha w + (1-\alpha)u) + \frac{\lambda}{2}||\alpha w + (1-\alpha) u||^2 + \frac{\lambda}{2}\alpha(1-\alpha)||w-u||^2 \\
        &\leq f(\alpha w + (1-\alpha)u) + \frac{\lambda}{2}(\alpha||w||^2 + (1-\alpha)||u||^2 - \alpha(1-\alpha)||w-u||^2) + \\
        &\hphantom{\leq} \quad \frac{\lambda}{2}\alpha(1-\alpha)||w-u||^2 \\
        &\leq f(\alpha w + (1-\alpha)u) + \frac{\lambda}{2}(\alpha||w||^2 + (1-\alpha)||u||^2) \\
        &\leq \alpha (f(w) + \frac{\lambda}{2}||w||^2) + (1-\alpha)(f(u) + \frac{\lambda}{2}||u||^2) \\
        &= \alpha g(w) + (1-\alpha)g(u)
      \end{align*}
        We conclude that $g$ is $\lambda$-strongly convex.
      \item Suppose $f(w) = g(w) - \frac{\lambda}{2}||w||^2$ is convex. We will show that $g$ is $\lambda$-strongly convex:
        \begin{align*}
          &\hphantom{=} g(\alpha w + (1-\alpha)u) + \frac{\lambda}{2}\alpha(1-\alpha)||w-u||^2 \\
          &= f(\alpha w + (1-\alpha)u) + \frac{\lambda}{2}||\alpha w + (1-\alpha) u||^2 + \frac{\lambda}{2}\alpha(1-\alpha)||w-u||^2 \\
          &= f(\alpha w + (1-\alpha)u) + \frac{\lambda}{2}(\alpha w + (1-\alpha) u)^T(\alpha w + (1-\alpha) u) + \\
          &\hphantom{=} \quad \frac{\lambda}{2}\alpha(1-\alpha)(w-u)^T(w-u) \\
          &= f(\alpha w + (1-\alpha)u) + \frac{\lambda}{2}\left(\alpha^2||w||^2+2\alpha(1-\alpha)(u\cdot w) + (1-\alpha)^2||u||^2\right) + \\
          &\hphantom{=} \quad\frac{\lambda}{2}\alpha(1-\alpha)\left(||w||^2 - 2(u\cdot w) + ||u||^2\right) \\
          &= f(\alpha w + (1-\alpha)u) + \frac{\lambda}{2}\left(\alpha^2||w||^2 + (1-\alpha)^2||u||^2\right) + \frac{\lambda}{2}\alpha(1-\alpha)\left(||w||^2 + ||u||^2\right) \\
          &= f(\alpha w + (1-\alpha)u) + \frac{\lambda}{2}||w||^2\left(\alpha^2 + \alpha(1-\alpha)\right) + \frac{\lambda}{2}||u||^2\left((1-\alpha)^2 + \alpha(1-\alpha)\right) \\
          &= f(\alpha w + (1-\alpha)u) + \frac{\lambda\alpha}{2}||w||^2 + \frac{\lambda(1-\alpha)}{2}||u||^2 \\
          &\leq \alpha f(w) + (1-\alpha)f(u) + \frac{\lambda\alpha}{2}||w||^2 + \frac{\lambda(1-\alpha)}{2}||u||^2 \\
          &\leq \alpha (f(w) + \frac{\lambda}{2}||w||^2) + (1-\alpha)(f(u) + \frac{\lambda}{2}||u||^2) \\
          &= \alpha g(w) + (1-\alpha)g(u)
        \end{align*}
        as desired.

        Now we must prove the converse. To that aim, assume $g$ is $\lambda$-strongly convex. We will now prove that $f$ is convex.
        \begin{align*}
          &\hphantom{=} f(\alpha w + (1-\alpha)u) \\
          &= g(\alpha w + (1-\alpha)u) - \frac{\lambda}{2}||\alpha w + (1-\alpha)u||^2 \\
          &\leq \alpha g(w) + (1-\alpha)g(u) - \frac{\lambda}{2}||\alpha w + (1-\alpha) u||^2 - \frac{\lambda}{2}\alpha(1-\alpha)||w-u||^2 \\
          &\leq \alpha g(w) + (1-\alpha)g(u) - \frac{\lambda}{2}(\alpha w + (1-\alpha) u)^T(\alpha w + (1-\alpha) u) - \\
          &\hphantom{\leq} \frac{\lambda}{2}\alpha(1-\alpha)(w-u)^T(w-u) \\
          &\leq \alpha g(w) + (1-\alpha)g(u) - \frac{\lambda}{2}\left(\alpha^2 ||w||^2 + 2\alpha(1-\alpha)(u \cdot w) + (1-\alpha) ||u||^2\right) - \\
          &\hphantom{\leq} \frac{\lambda}{2}\alpha(1-\alpha)(||w||^2 - 2(u\cdot w) + ||u||^2) \\
          &\leq \alpha g(w) + (1-\alpha)g(u) - \frac{\lambda}{2}\left(\alpha^2 ||w||^2 + (1-\alpha) ||u||^2\right) - \frac{\lambda}{2}\alpha(1-\alpha)(||w||^2 + ||u||^2) \\
          &\leq \alpha g(w) - \frac{\lambda}{2}||w||^2\left(\alpha^2 + \alpha(1-\alpha) \right)  + (1-\alpha)g(u) - \frac{\lambda}{2}||u||^2\left((1-\alpha)^2 + \alpha(1-\alpha)\right) \\
          &\leq \alpha g(w) - \frac{\lambda\alpha}{2}||w||^2  + (1-\alpha)g(u) - \frac{\lambda(1-\alpha)}{2}||u||^2 \\
          &\leq \alpha \left(g(w) - \frac{\lambda}{2}||w||^2\right) + (1-\alpha)\left(g(u) - \frac{\lambda}{2}||u||^2\right) \\
          &= \alpha f(w) + (1-\alpha)f(u)
        \end{align*}
        so $f$ is convex.
  \end{enumerate}
\end{solution}

\begin{problem}{Exercise 3.6 (Random Variables and Strong Convexity)}
  Let $X \in \mathbb{R}$ be a discrete vector-valued random variable, with distribution
  \[
  \rho(X) =
  \begin{cases}
    w, & \text{with probability }\alpha \\
    u, & \text{with probability }(1 - \alpha)
  \end{cases}
  \]
  \begin{enumerate}
      \item Find $E[X]^2$, $E[X^2]$, $Var(X)$. Show that the expression $E[X^2] - (E[X])^2 = Var(X)$ corresponds to
      \[
      \alpha f(w) + (1 - \alpha) f(u) - f(\alpha w + (1 - \alpha) u) = \alpha (1 - \alpha) \|w - u\|^2
      \]
      with $f(w) = w^2$.

      \item Suppose $f$ is $\lambda$-strongly convex. Let $X$ be as above. Show that
      \[
      E[f(X)] - f(E[X]) \geq \frac{\lambda}{2} Var(X)
      \]
  \end{enumerate}
\end{problem}

\begin{solution}
  \begin{enumerate}
    \item
      \begin{enumerate}
        \item $E[X] = \alpha w + (1-\alpha)u$ therefore $E[X]^2 = ||\alpha w + (1-\alpha)u||^2$
        \item $E[X^2] = \alpha||w||^2 + (1-\alpha) ||u||^2$
        \item $\begin{aligned}[t]
          Var[X] &= E[X^2] - E[X]^2 \\
          &= \alpha||w||^2 + (1-\alpha) ||u||^2 - ||\alpha w + (1-\alpha)u||^2 \\
          &= \alpha||w||^2 + (1-\alpha) ||u||^2 - (\alpha w + (1-\alpha)u)^T(\alpha w + (1-\alpha)u) \\
          &= \alpha||w||^2 + (1-\alpha) ||u||^2 - \left(\alpha^2||w||^2 -2\alpha(1-\alpha)(u\cdot w) + (1-\alpha)^2||u||^2\right) \\
          &= ||w||^2(\alpha -\alpha^2)+2\alpha(1-\alpha)(u\cdot w) + ||u||^2(1-\alpha-(1-\alpha)^2) \\
          &= ||w||^2\alpha(1 -\alpha)+2\alpha(1-\alpha)(u\cdot w) + ||u||^2\alpha(1 - \alpha) \\
          &= \alpha(1-\alpha)\left(||w||^2 - 2(u\cdot w) + ||u||^2\right) \\
          &= \alpha(1-\alpha)||w-u||^2
          \end{aligned}$
      \end{enumerate}
    \item $\begin{aligned}[t]
      E[f(X)] - f(E[X]) &= \alpha f(w) + (1-\alpha)f(u) - f(\alpha w + (1-\alpha)u) \\
      &\geq \alpha f(w) + (1-\alpha)f(u) - \alpha f(w) -(1-\alpha)f(u) + \frac{\lambda}{2}\alpha(1-\alpha)||w-u||^2 \\
      &\geq \frac{\lambda}{2}\alpha(1-\alpha)||w-u||^2 = \frac{\lambda}{2}Var[x]
      \end{aligned}$
  \end{enumerate}
\end{solution}

\begin{problem}{Exercise 3.7 (Algorithmic Stability for Ridge Regression)}
Write,
    \[
    f(w, S) = \hat{L}_S(w) = \frac{1}{m} \sum_{i=1}^{m} (w \cdot x_i - y_i)^2
    \]
    Consider the datasets $S_1$, $S_2$, which differ by exactly one data point. Write
    \[
    z_1 = (x_1, y_1) \in S_1, \quad z_2 = (x_2, y_2) \in S_2
    \]
    for the different points. Define $g(w, S) = f(w, S) + \frac{\lambda}{2} \|w\|^2$ and
    \[
    w_i = A(S_i) = \arg\min_w g(w, S_i), \quad i = 1, 2
    \]
    \begin{enumerate}
        \item Explain why $g(w, S)$ is $\lambda$-strongly convex (as a function of $w$).

        \item Explain how strong convexity is applied to obtain
        \[
        g(w_2, S_1) - g(w_1, S_1) \geq \frac{\lambda}{2} \|w_1 - w_2\|^2
        \]
        \[
        g(w_1, S_2) - g(w_2, S_2) \geq \frac{\lambda}{2} \|w_1 - w_2\|^2
        \]

        \item Simplify to obtain
        \[
        \lambda \|w_1 - w_2\|^2 \leq \frac{1}{m} | \ell(w_2, z_1) - \ell(w_1, z_1) + \ell(w_1, z_2) - \ell(w_2, z_2) |
        \]
        for $\ell(w, z) = (w \cdot x - y)^2$.

        \item Assume that $\|x\| \leq C_x$ and $|y| \leq C_y$ for all data $x$, $y$, and that $\|w_1\|, \|w_2\| \leq C_w$. Bound the RHS of the previous inequality. (Hint: apply the identity $(a^2 - b^2) = (a - b)(a + b)$ to the loss.)

        \item Conclude that the Ridge Regression problem is replace one stable (ROS) and give the rate as a function of $\lambda$, $m$.
    \end{enumerate}
\end{problem}

\begin{solution}
  \begin{enumerate}
    \item $g$ is $\lambda$-strongly convex by the proof in exercise 3.5(c)
    \item Since $g$ is $\lambda$-strongly convex, we apply the inequality setting $\alpha=0.5$): $$g(\frac{1}{2} w_1 + \frac{1}{2}w_2, S_1) + \frac{\lambda}{4}||w_1-w_2||^2 \leq \frac{1}{2} g(w_1, S_1) + \frac{1}{2}g(w_2, S_1)$$

      Because $w_1$ minimises $g$ at $S_1$, we have that $g(w_1, S_1) \leq g(\frac{1}{2} w_1 + \frac{1}{2} w_2, S_1)$. So we can substitute $g(w_1, S_1)$ in the left hand side in the following way: $$g(w_1, S_1) + \frac{\lambda}{4}||w_1-w_2||^2 \leq \frac{1}{2}g(w_1, s_1) + \frac{1}{2}g(w_2, S_1)$$

      Subtracting $g(w_1, S_1)$ from both sides of the inequality, and then multiplying both sides by 2 gives the first desired inequality:
      $$\frac{\lambda}{2}||w_1-w_2||^2 \leq g(w_2,S_1)-g(w_1, S_1)$$
      We can obtain the other inequality by the same argument, or via symmetry.
      $$\frac{\lambda}{2}||w_1-w_2||^2 \leq g(w_1,S_2)-g(w_2, S_2)$$
    \item We start by rewriting the inequalities from the preceding exercise.
      \begin{align*}
        \frac{\lambda}{2}||w_1-w_2||^2 \leq g(w_2, z_1)-g(w_1, z_1) \\
        \frac{\lambda}{2}||w_1-w_2||^2 \leq g(w_1, z_2)-g(w_2, z_2)
      \end{align*}
      Then we add both inequalities together to obtain:
      \begin{align*}
        \lambda ||w_1-w_2||^2 &\leq (g(w_2,z_1)-g(w_1,z_1)+g(w_1,z_2)-g(w_2,z_2)) \\
        \hphantom{\lambda ||w_1-w_2||^2} &\leq (f(w_2,z_1) + ||w_2||^2 -f(w_1,z_1) - ||w_2||^2 \\
        \hphantom{\lambda ||w_1-w_2||^2}&+f(w_1,z_2) + ||w_1||^2-f(w_2,z_2) - ||w_1||^2 \\
        &\leq f(w_2,z_1)-f(w_1,z_1)+f(w_1,z_2)-f(w_2,z_2) \\
        &\leq \frac{1}{m} (\ell(w_2,z_1)-\ell(w_1,z_1)+\ell(w_1,z_2)-\ell(w_2,z_2))
      \end{align*}
      where the last inequality comes from the fact that $f(w_i,S_i) = \frac{1}{m}\ell(w_i, z_i)$.
    \item We begin by applying the hint with $\ell(w_2,z_1) - \ell(w_1,z_1)$:
        \begin{align*}
          \ell(w_2,z_1) - \ell(w_1,z_1) &= (w_2\cdot x - y)^2 - (w_1\cdot x - y)^2 \\
          \hphantom{\ell(w_2,z_1) - \ell(w_1,z_1)} &= (w_2\cdot x - y - w_1\cdot x + y)(w_2\cdot x - y + w_1\cdot x - y) \\
          \hphantom{\ell(w_2,z_1) - \ell(w_1,z_1)} &= ((w_2 - w_1)\cdot x)((w_1 + w_2)\cdot x - 2y) \\
          \hphantom{\ell(w_2,z_1) - \ell(w_1,z_1)} &\leq (2C_wC_x)(2C_wC_x+2C_y) \\
          \hphantom{\ell(w_2,z_1) - \ell(w_1,z_1)} &\leq 2C_wC_x(C_wC_x + C_y)
        \end{align*}
        Now we can apply the triangle inequality to the right hand side of the inequality from the previous exercise, and then we will apply our upper bound.
        \begin{align*}
          &\hphantom{\leq} \frac{1}{m} |\ell(w_2,z_1)-\ell(w_1,z_1)+\ell(w_1,z_2)-\ell(w_2,z_2)| \\
          &\leq \frac{1}{m} |\ell(w_2,z_1)-\ell(w_1,z_1)| + \frac{1}{m}|\ell(w_1,z_2)-\ell(w_2,z_2)| \\
          &\leq \frac{4C_wC_x(C_wC_x + C_y)}{m}
        \end{align*}
      \item Since $L(A(Z_1)) - L(A(Z_2))$ can be bounded by function of $m$, the Ridge Regression problem is replace-one-stable (ROS).

        The ROS rate, denoted $\beta(m)$ is given by the following:
        \[
        \begin{aligned}[t]
        L(A(Z_1))-L(A(Z_2)) &= \frac{1}{m} |\ell(w_2,z_1)-\ell(w_1,z_1)+\ell(w_1,z_2)-\ell(w_2,z_2)| \\
        &\leq \frac{4C_wC_x(C_wC_x + C_y)}{m} = \beta(m)
        \end{aligned}
        \]
  \end{enumerate}
\end{solution}

%----------------------------------------------------------------------------------------
%	BIBLIOGRAPHY
%----------------------------------------------------------------------------------------

%% \bibliography{sample.bib} % The file containing the bibliography

\end{document}
