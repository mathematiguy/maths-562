%%%%%%%%%%%%%%%%%%%%%%%%%%%%%%%%%%%%%%%%%
% Arsclassica Article
% LaTeX Template
% Version 1.1 (1/8/17)
%
% This template has been downloaded from:
% http://www.LaTeXTemplates.com
%
% Original author:
% Lorenzo Pantieri (http://www.lorenzopantieri.net) with extensive modifications by:
% Vel (vel@latextemplates.com)
%
% License:
% CC BY-NC-SA 3.0 (http://creativecommons.org/licenses/by-nc-sa/3.0/)
%
%%%%%%%%%%%%%%%%%%%%%%%%%%%%%%%%%%%%%%%%%

%----------------------------------------------------------------------------------------
%	PACKAGES AND OTHER DOCUMENT CONFIGURATIONS
%----------------------------------------------------------------------------------------

\documentclass[
	10pt, % Main document font size
	a4paper, % Paper type, use 'letterpaper' for US Letter paper
	oneside, % One page layout (no page indentation)
	%twoside, % Two page layout (page indentation for binding and different headers)
	headinclude,footinclude, % Extra spacing for the header and footer
	BCOR5mm, % Binding correction
]{scrartcl}
\renewcommand{\labelenumii}{\theenumii}
\renewcommand{\theenumii}{\theenumi.\arabic{enumii}.}
\usepackage{mdframed}
\usepackage{amsmath}
\DeclareMathOperator*{\argmax}{argmax} % thin space, limits underneath in displays
\DeclareMathOperator*{\argmin}{argmin} % thin space, limits underneath in displays
\DeclareMathOperator*{\sig}{sig}

%%%%%%%%%%%%%%%%%%%%%%%%%%%%%%%%%%%%%%%%%
% Arsclassica Article
% Structure Specification File
%
% This file has been downloaded from:
% http://www.LaTeXTemplates.com
%
% Original author:
% Lorenzo Pantieri (http://www.lorenzopantieri.net) with extensive modifications by:
% Vel (vel@latextemplates.com)
%
% License:
% CC BY-NC-SA 3.0 (http://creativecommons.org/licenses/by-nc-sa/3.0/)
%
%%%%%%%%%%%%%%%%%%%%%%%%%%%%%%%%%%%%%%%%%

%----------------------------------------------------------------------------------------
%	REQUIRED PACKAGES
%----------------------------------------------------------------------------------------

\usepackage[
nochapters, % Turn off chapters since this is an article        
beramono, % Use the Bera Mono font for monospaced text (\texttt)
eulermath,% Use the Euler font for mathematics
pdfspacing, % Makes use of pdftex’ letter spacing capabilities via the microtype package
dottedtoc % Dotted lines leading to the page numbers in the table of contents
]{classicthesis} % The layout is based on the Classic Thesis style

\usepackage{arsclassica} % Modifies the Classic Thesis package

\usepackage[T1]{fontenc} % Use 8-bit encoding that has 256 glyphs

\usepackage[utf8]{inputenc} % Required for including letters with accents

\usepackage{graphicx} % Required for including images
\graphicspath{{Figures/}} % Set the default folder for images

\usepackage{enumitem} % Required for manipulating the whitespace between and within lists

\usepackage{lipsum} % Used for inserting dummy 'Lorem ipsum' text into the template

\usepackage{subfig} % Required for creating figures with multiple parts (subfigures)

\usepackage{amsmath,amssymb,amsthm} % For including math equations, theorems, symbols, etc

\usepackage[round]{natbib}   % omit 'round' option if you prefer square brackets

\usepackage{varioref} % More descriptive referencing

\bibliographystyle{plainnat}

%----------------------------------------------------------------------------------------
%	THEOREM STYLES
%---------------------------------------------------------------------------------------

\theoremstyle{definition} % Define theorem styles here based on the definition style (used for definitions and examples)
\newtheorem{definition}{Definition}

\theoremstyle{plain} % Define theorem styles here based on the plain style (used for theorems, lemmas, propositions)
\newtheorem{theorem}{Theorem}

\theoremstyle{remark} % Define theorem styles here based on the remark style (used for remarks and notes)

%----------------------------------------------------------------------------------------
%	HYPERLINKS
%---------------------------------------------------------------------------------------

\hypersetup{
%draft, % Uncomment to remove all links (useful for printing in black and white)
colorlinks=true, breaklinks=true, bookmarks=true,bookmarksnumbered,
urlcolor=webbrown, linkcolor=RoyalBlue, citecolor=webgreen, % Link colors
pdftitle={}, % PDF title
pdfauthor={\textcopyright}, % PDF Author
pdfsubject={}, % PDF Subject
pdfkeywords={}, % PDF Keywords
pdfcreator={pdfLaTeX}, % PDF Creator
pdfproducer={LaTeX with hyperref and ClassicThesis} % PDF producer
}
 % Include the structure.tex file which specified the document structure and layout

\hyphenation{Fortran hy-phen-ation} % Specify custom hyphenation points in words with dashes where you would like hyphenation to occur, or alternatively, don't put any dashes in a word to stop hyphenation altogether

%----------------------------------------------------------------------------------------
%	TITLE AND AUTHOR(S)
%----------------------------------------------------------------------------------------

%\subtitle{Subtitle} % Uncomment to display a subtitle

\author{\spacedlowsmallcaps{Caleb Moses*}} % The article author(s) - author affiliations need to be specified in the AUTHOR AFFILIATIONS block

\date{} % An optional date to appear under the author(s)


\newenvironment{problem}[2][]
{ \begin{mdframed}[backgroundcolor=gray!20] \textbf{#1 #2} \\}
		{  \end{mdframed}}

% Define solution environment
\newenvironment{solution}
{\textit{Solution:}}
{}

%----------------------------------------------------------------------------------------

\title{MATH/COMP 562 ASSIGNMENT 2}
\author{Caleb Moses}

\begin{document}

\maketitle

\section{Rademacher Complexity}

\begin{problem}{Exercise 1.1 (Rademacher complexity of linear hypothesis)}
Prove the following theorem.

\textbf{(Mohri) Theorem 5.10:} Define $B_r = \{x \in \mathbb{R}^d: ||x|| \leq r\}$. Let $S = \{x_1 , \ldots, x_m\} \subset X \subset \mathbb{R}^d$ with $X = B_r$. Consider the linear hypotheses $h(w, x) = w \cdot x$ and set $\mathcal{H} = \{h(x, w) = w \cdot x: x \in X, w \in B_\lambda\}$. Prove that the empirical Rademacher complexity is bounded as follows,
\[
	\hat{\mathfrak{R}}_S(H) \leq \frac{r\lambda}{\sqrt{m}}
\]
Hint: refer to class notes, Mohri textbook Theorem 5.10.
\end{problem}

\begin{solution}
	\begin{align}
		\hat{\mathfrak{R}}_S(\mathcal{H}) & = \frac{1}{m}\mathbb{E}_\sigma\left[ \sup_{||w||\leq \lambda}\sum_{i=1}^m\sigma_i w \cdot x_i \right] \text{ by definition}                                                                               \\
		                                  & = \frac{1}{m}\mathbb{E}_\sigma\left[ \sup_{||w||\leq \lambda} w\cdot \sum_{i=1}^m \sigma_i x_i \right] \text{ by linearity of the dot product}                                                            \\
		                                  & \leq \frac{1}{m}\mathbb{E}_\sigma\left[ \lambda \left|\left|\sum_{i=1}^m \sigma_i x_i \right|\right|\right] \text{ applying Cauchy-Schwartz and $||w|| \leq \lambda$}                                     \\
		                                  & \leq \frac{\lambda}{m}\mathbb{E}_\sigma\left[ \left|\left|\sum_{i=1}^m \sigma_i x_i \right|\right|\right] \text{ by the linearity of expectation}                                                         \\
		                                  & \leq \frac{\lambda}{m}\mathbb{E}_\sigma\left[ {\left(\left|\left|\sum_{i=1}^m \sigma_i x_i \right|\right|^2\right)}^\frac{1}{2}\right] \text{ since $||x|| = \sqrt{||x||^2}$}                             \\
		                                  & = \frac{\lambda}{m}\mathbb{E}_\sigma\left[ {\left(\sum_{i,j=1}^m \sigma_j \sigma_j (x_i \cdot x_j)\right)}^{\frac{1}{2}} \right] \text{since $||x||^2 = x^T x$ and rearranging terms}                     \\
		                                  & \leq \frac{\lambda}{m}{\left[\sum_{i=1}^m ||x_i||^2\right]}^{\frac{1}{2}} \text{ since $\mathbb{E}_\sigma[\sigma_i \sigma_j] = \mathbb{E}_\sigma[\sigma_i]\mathbb{E}_\sigma[\sigma_j] = 0$ for $i\neq j$} \\
		                                  & \leq \frac{\lambda \sqrt{mr^2}}{m} = \frac{r\lambda}{m} \text{ as required.}
	\end{align}
	Note that we are able to remove the expectation in line (7) since the dependency on $\sigma$ is removed.
\end{solution}

\begin{problem}{Exercise 1.2}
Given a hypothesis class $\mathcal{H}$, of functions $h : X \rightarrow \mathbb{R}$, and a dataset $S = \{x_1 , \ldots, x_m\} \subset X$. Define $\Phi(S) = \sup_{h\in \mathcal{H}} |\mathbb{E}[h] - \hat{\mathbb{E}}_S[h]|$ to be the least upper bound for the generalization gap of a function in $\mathcal{H}$.

Prove that $\hat{E}_S [\Phi(S)] \leq 2\hat{\mathfrak{R}}_m(\mathcal{H})$.
\end{problem}

\begin{solution}
	Let $S$ and $S^\prime$ be two datasets differing in one point, say $z_m\in S$ and $z_m^\prime \in S^\prime$. We proceed by bounding the expectation $\hat{\mathbb{E}}_S[\Phi(S)]$ as follows:
	\begin{align*}
		\hat{\mathbb{E}}_S[\Phi(S)] & = \hat{\mathbb{E}}_S\left[ \sup_{h\in\mathcal{H}} \left| \mathbb{E}[h] - \hat{\mathbb{E}}_S[h] \right| \right]                                                                                       \\
		                            & = \mathbb{E}_S \left[ \sup_{h\in\mathcal{H}} \hat{\mathbb{E}}_{S^\prime}[h] - \hat{\mathbb{E}}_{S}[h] \right]                                                                                        \\
		                            & \leq \mathbb{E}_{S,S^\prime}\left[ \sup_{h\in\mathcal{H}} \hat{\mathbb{E}}_{S^\prime}[h] - \hat{\mathbb{E}}_{S}[h] \right]                                                                           \\
		                            & = \mathbb{E}_{S,S^\prime} \left[ \sup_{h\in\mathcal{H}} \frac{1}{m} \sum_{i=1}^m\left(h(z_i^\prime) - h(z_i)\right) \right]                                                                          \\
		                            & = \mathbb{E}_{\sigma, S, S^\prime} \left[ \sup_{h\in\mathcal{H}} \frac{1}{m} \sum_{i=1}^m \sigma_i (h(z_i^\prime) - h(z_i)) \right]                                                                  \\
		                            & \leq \mathbb{E}_{\sigma, S^\prime}\left[ \sup_{h\in\mathcal{H}} \frac{1}{m} \sigma_i h(z_i^\prime) \right] + \mathbb{E}_{\sigma, S}\left[ \sup_{h\in\mathcal{H}} \frac{1}{m} \sigma_i h(z_i) \right] \\
		                            & = 2\mathbb{E}_{\sigma,S}\left[ \sup_{h\in\mathcal{H}} \frac{1}{m} \sum_{i=1}^m \sigma h(z_i) \right] = 2\hat{\mathfrak{R}}_m(\mathcal{H})
	\end{align*}
\end{solution}

\begin{problem}{Exercise 1.3 (Rademacher Identities)}
Fix $m\geq 1$. Prove the following identities for any $\alpha \in \mathbb{R}$ and any two hypothesis sets $\mathcal{H}$ and $\mathcal{H^\prime}$ of functions mapping from $\mathcal{X}$ to $\mathcal{R}$.
\begin{enumerate}[label= (\alph*)]
	\item $\mathfrak{R}_m(\alpha \mathcal{H}) = |\alpha| \mathfrak{R}_m(\mathcal{H})$.
	\item $\mathfrak{R}_m(\mathcal{H} + \mathcal{H}^\prime) = \mathfrak{R}_m(\mathcal{H}) + \mathfrak{R}_m(\mathcal{H^\prime})$.
\end{enumerate}
\end{problem}

\begin{solution}
	\begin{enumerate}[label= (\alph*)]
		\item We can bound $\mathfrak{R}_m(\alpha \mathcal{H})$ above and below and establish our result using the squeeze theorem.
		      \begin{align*}
			      \mathbb{E}_\sigma \left[ \sup_{h\in\mathcal{H}} \frac{1}{m} \sum_{i=1}^m \sigma_i (-|\alpha| h(z_i)) \right]                & \leq \mathfrak{R}_m(\alpha \mathcal{H}) & \leq \mathbb{E}_\sigma \left[ \sup_{h\in\mathcal{H}} \frac{1}{m} \sum_{i=1}^m \sigma_i (|\alpha| h(z_i)) \right] \\
			      \mathbb{E}_{\sigma^\prime} \left[ \sup_{h\in\mathcal{H}} \frac{1}{m} \sum_{i=1}^m \sigma_i^\prime (|\alpha| h(z_i)) \right] & \leq \mathfrak{R}_m(\alpha \mathcal{H}) & \leq \mathbb{E}_\sigma \left[ \sup_{h\in\mathcal{H}} \frac{1}{m} \sum_{i=1}^m \sigma_i (|\alpha| h(z_i)) \right]
		      \end{align*}
		      Both sides of the last inequality are equal to each other since $\sigma$ and $\sigma^\prime = -\sigma$ are identically distributed. Lastly we use linearity of expectation and the independence of $\alpha$ on $h$ to shift the $|\alpha|$ term outside of the expectation in the following way:
		      \begin{align*}
			      \mathfrak{R}_m(\alpha \mathcal{H}) & = \mathbb{E}_\sigma \left[ \sup_{h\in\mathcal{H}} \frac{1}{m} \sum_{i=1}^m \sigma_i (|\alpha| h(z_i)) \right] & = |\alpha| \mathbb{E}_\sigma \left[ \sup_{h\in\mathcal{H}} \frac{1}{m} \sum_{i=1}^m \sigma_i h(z_i) \right] \\
			                                         & = |\alpha| \mathfrak{R}_m(\mathcal{H}) \text{ this concludes the proof}
		      \end{align*}
		      \newpage
		\item We will rely on the fact that $h$ and $h^\prime$ do not depend on one another so maximising the sum is the same as maximising each separately. As a result, we can use the following argument: \begin{align*}
			      \mathfrak{R}_m(\mathcal{H} + \mathcal{H^\prime}) & = \mathbb{E}_\sigma \left[ \sup_{h\in\mathcal{H}, h^\prime \in \mathcal{H}^\prime} \frac{1}{m} \sum_{i=1}^m \sigma_i (h(z_i) + h^\prime (z_i)) \right]                                                                           \\
			                                                       & = \mathbb{E}_\sigma \left[ \sup_{h\in\mathcal{H}} \frac{1}{m} \sum_{i=1}^m \sigma_i (h(z_i)) \right] + \mathbb{E}_\sigma \left[ \sup_{h^\prime \in \mathcal{H}^\prime} \frac{1}{m} \sum_{i=1}^m \sigma_i (h^\prime(z_i)) \right] \\
			                                                       & = \mathfrak{R}_m(\mathcal{H}) + \mathfrak{R}_m(\mathcal{H}^\prime)
		      \end{align*}

	\end{enumerate}
\end{solution}
\section{Convex Learning Problems}
Refer to Ch 12 of Understanding Machine Learning (Shalev-Shwartz).

\begin{problem}{Exercise 1.4 (non convexity of 0\textendash{}1 loss)}
Construct an example showing that the 0\textendash{}1 loss function may suffer from local minima; namely, construct a training sample $S \in {(X \times \{\pm1\})}^m$ (say, for $X = \mathbb{R}^2$), for which there exist a vector $w$ and some $\varepsilon > 0$ such that:
\begin{enumerate}

	\item For any $w'$ such that $\|w - w'\| \leq \varepsilon$ we have $L_S (w) \leq L_S (w')$ (where the loss here is the 0\textendash{}1 loss). This means that $w$ is a local minimum of $L_S$.
	\item There exists some $w^*$ such that $L_S (w^*) < L_S (w)$. This means that $w$ is not a global minimum of $L_S$.
\end{enumerate}

Hint: Consider samples in a checkerboard pattern.
\end{problem}

\begin{solution}

\end{solution}

\begin{problem}{Exercise 1.5 (Convexity, Lipschitz, and Smoothness of logistic regression loss)}
Consider the learning problem of logistic regression: Let $H = X = \{x \in \mathbb{R}^d : \|x\| \leq B\}$, for some scalar $B > 0$, let $Y = \{\pm1\}$, and let the loss function $\ell$ be defined as $\ell(w, (x, y)) = \log(1 + \exp(-y\langle w, x \rangle))$. Show that the resulting learning problem is both convex-Lipschitz-bounded and convex-smooth-bounded. Specify the parameters of Lipschitzness and smoothness.
\end{problem}

\begin{solution}

\end{solution}


\begin{problem}{Exercise 1.6 (Lipschitz continuity of the hinge loss)}
Consider the problem of learning halfspaces with the hinge loss. We limit our domain to the Euclidean ball with radius $R$. That is, $X = \{x : \|x\|_2 \leq R\}$. The label set is $Y = \{\pm1\}$ and the loss function $\ell$ is defined by $\ell(w, (x, y)) = \max\{0, 1 - y\langle w, x \rangle\}$. We already know that the loss function is convex. Show that it is $R$-Lipschitz.
\end{problem}

\begin{solution}

\end{solution}

%----------------------------------------------------------------------------------------
%	BIBLIOGRAPHY
%----------------------------------------------------------------------------------------

%% \bibliography{sample.bib} % The file containing the bibliography

\end{document}
