%%%%%%%%%%%%%%%%%%%%%%%%%%%%%%%%%%%%%%%%%
% Arsclassica Article
% LaTeX Template
% Version 1.1 (1/8/17)
%
% This template has been downloaded from:
% http://www.LaTeXTemplates.com
%
% Original author:
% Lorenzo Pantieri (http://www.lorenzopantieri.net) with extensive modifications by:
% Vel (vel@latextemplates.com)
%
% License:
% CC BY-NC-SA 3.0 (http://creativecommons.org/licenses/by-nc-sa/3.0/)
%
%%%%%%%%%%%%%%%%%%%%%%%%%%%%%%%%%%%%%%%%%

%----------------------------------------------------------------------------------------
%	PACKAGES AND OTHER DOCUMENT CONFIGURATIONS
%----------------------------------------------------------------------------------------

\documentclass[
	10pt, % Main document font size
	a4paper, % Paper type, use 'letterpaper' for US Letter paper
	oneside, % One page layout (no page indentation)
	%twoside, % Two page layout (page indentation for binding and different headers)
	headinclude,footinclude, % Extra spacing for the header and footer
	BCOR5mm, % Binding correction
]{scrartcl}
\renewcommand{\labelenumii}{\theenumii}
\renewcommand{\theenumii}{\theenumi.\arabic{enumii}.}
\usepackage{mdframed}
\usepackage{amsmath}
\DeclareMathOperator*{\argmax}{argmax} % thin space, limits underneath in displays
\DeclareMathOperator*{\argmin}{argmin} % thin space, limits underneath in displays
\DeclareMathOperator*{\sig}{sig}

\input{structure.tex} % Include the structure.tex file which specified the document structure and layout

\hyphenation{Fortran hy-phen-ation} % Specify custom hyphenation points in words with dashes where you would like hyphenation to occur, or alternatively, don't put any dashes in a word to stop hyphenation altogether

%----------------------------------------------------------------------------------------
%	TITLE AND AUTHOR(S)
%----------------------------------------------------------------------------------------

%\subtitle{Subtitle} % Uncomment to display a subtitle

\author{\spacedlowsmallcaps{Caleb Moses*}} % The article author(s) - author affiliations need to be specified in the AUTHOR AFFILIATIONS block

\date{} % An optional date to appear under the author(s)


\newenvironment{problem}[2][]
{ \begin{mdframed}[backgroundcolor=gray!20] \textbf{#1 #2} \\}
		{  \end{mdframed}}

% Define solution environment
\newenvironment{solution}
{\textit{Solution:}}
{}

%----------------------------------------------------------------------------------------

\title{MATH/COMP 562 ASSIGNMENT 2}
\author{Caleb Moses}

\begin{document}

gin{document}

\maketitle

\section{Rademacher Complexity}

\begin{problem}{Exercise 1.1 (Rademacher complexity of linear hypothesis)}
Prove the following theorem.

\textbf{(Mohri) Theorem 5.10:} Define $B_r = \{x \in \mathbb{R}^d | ||x|| \leq r\}$. Let $S = \{x_1 , \ldots, x_m\} \subset X \subset \mathbb{R}^d$ with $X = B_r$. Consider the linear hypotheses $h(w, x) = w \cdot x$ and set $H = \{h(x, w) = w \cdot x | x \in X, w \in B_\lambda\}$. Prove that the empirical Rademacher complexity is bounded as follows,
\[
	\hat{R}_S(H) \leq \frac{r\lambda}{\sqrt{m}}
\]
Hint: refer to class notes, Mohri textbook Theorem 5.10.
\end{problem}

\begin{solution}
	Here is a solution
\end{solution}

\subsection*{Exercise 1.2}
Given a hypothesis class $H$, of functions $h : X \rightarrow \mathbb{R}$, and a dataset $S = \{x_1 , \ldots, x_m\} \subset X$. Define $\Phi(S) = \sup_{h\in H} |E[h] - \hat{E}_S[h]|$ to be the least upper bound for the generalization gap of a function in $H$. Prove that $\hat{E}_S [\Phi(S)] \leq 2\hat{R}_m(H)$.

\subsection*{Exercise 1.3 (Rademacher Identities)}
Mohri 3.8(a) and 3.8(b).

\section{Convex Learning Problems}
Refer to Ch 12 of Understanding Machine Learning (Shalev-Shwartz).

\subsection*{Exercise 1.4 (non convexity of 0\-1 loss)}
Problem 12.1. Hint: Consider samples in a checkerboard pattern.

\subsection*{Exercise 1.5 (Convexity, Lipschitz, and Smoothness of logistic regression loss)}
Problem 12.2.

\subsection*{Exercise 1.6 (Lipschitz continuity of the hinge loss)}
Problem 12.3.

%----------------------------------------------------------------------------------------
%	BIBLIOGRAPHY
%----------------------------------------------------------------------------------------

%% \bibliography{sample.bib} % The file containing the bibliography

\end{document}
