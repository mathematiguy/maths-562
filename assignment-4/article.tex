%%%%%%%%%%%%%%%%%%%%%%%%%%%%%%%%%%%%%%%%%
% Arsclassica Article
% LaTeX Template
% Version 1.1 (1/8/17)
%
% This template has been downloaded from:
% http://www.LaTeXTemplates.com
%
% Original author:
% Lorenzo Pantieri (http://www.lorenzopantieri.net) with extensive modifications by:
% Vel (vel@latextemplates.com)
%
% License:
% CC BY-NC-SA 3.0 (http://creativecommons.org/licenses/by-nc-sa/3.0/)
%
%%%%%%%%%%%%%%%%%%%%%%%%%%%%%%%%%%%%%%%%%

%----------------------------------------------------------------------------------------
%	PACKAGES AND OTHER DOCUMENT CONFIGURATIONS
%----------------------------------------------------------------------------------------

\documentclass[
10pt, % Main document font size
a4paper, % Paper type, use 'letterpaper' for US Letter paper
oneside, % One page layout (no page indentation)
%twoside, % Two page layout (page indentation for binding and different headers)
headinclude,footinclude, % Extra spacing for the header and footer
BCOR5mm, % Binding correction
]{scrartcl}

\usepackage{amsmath}
\DeclareMathOperator*{\argmax}{argmax} % thin space, limits underneath in displays
\DeclareMathOperator*{\argmin}{argmin} % thin space, limits underneath in displays
\DeclareMathOperator*{\sig}{sig}

%%%%%%%%%%%%%%%%%%%%%%%%%%%%%%%%%%%%%%%%%
% Arsclassica Article
% Structure Specification File
%
% This file has been downloaded from:
% http://www.LaTeXTemplates.com
%
% Original author:
% Lorenzo Pantieri (http://www.lorenzopantieri.net) with extensive modifications by:
% Vel (vel@latextemplates.com)
%
% License:
% CC BY-NC-SA 3.0 (http://creativecommons.org/licenses/by-nc-sa/3.0/)
%
%%%%%%%%%%%%%%%%%%%%%%%%%%%%%%%%%%%%%%%%%

%----------------------------------------------------------------------------------------
%	REQUIRED PACKAGES
%----------------------------------------------------------------------------------------

\usepackage[
nochapters, % Turn off chapters since this is an article        
beramono, % Use the Bera Mono font for monospaced text (\texttt)
eulermath,% Use the Euler font for mathematics
pdfspacing, % Makes use of pdftex’ letter spacing capabilities via the microtype package
dottedtoc % Dotted lines leading to the page numbers in the table of contents
]{classicthesis} % The layout is based on the Classic Thesis style

\usepackage{arsclassica} % Modifies the Classic Thesis package

\usepackage[T1]{fontenc} % Use 8-bit encoding that has 256 glyphs

\usepackage[utf8]{inputenc} % Required for including letters with accents

\usepackage{graphicx} % Required for including images
\graphicspath{{Figures/}} % Set the default folder for images

\usepackage{enumitem} % Required for manipulating the whitespace between and within lists

\usepackage{lipsum} % Used for inserting dummy 'Lorem ipsum' text into the template

\usepackage{subfig} % Required for creating figures with multiple parts (subfigures)

\usepackage{amsmath,amssymb,amsthm} % For including math equations, theorems, symbols, etc

\usepackage[round]{natbib}   % omit 'round' option if you prefer square brackets

\usepackage{varioref} % More descriptive referencing

\bibliographystyle{plainnat}

%----------------------------------------------------------------------------------------
%	THEOREM STYLES
%---------------------------------------------------------------------------------------

\theoremstyle{definition} % Define theorem styles here based on the definition style (used for definitions and examples)
\newtheorem{definition}{Definition}

\theoremstyle{plain} % Define theorem styles here based on the plain style (used for theorems, lemmas, propositions)
\newtheorem{theorem}{Theorem}

\theoremstyle{remark} % Define theorem styles here based on the remark style (used for remarks and notes)

%----------------------------------------------------------------------------------------
%	HYPERLINKS
%---------------------------------------------------------------------------------------

\hypersetup{
%draft, % Uncomment to remove all links (useful for printing in black and white)
colorlinks=true, breaklinks=true, bookmarks=true,bookmarksnumbered,
urlcolor=webbrown, linkcolor=RoyalBlue, citecolor=webgreen, % Link colors
pdftitle={}, % PDF title
pdfauthor={\textcopyright}, % PDF Author
pdfsubject={}, % PDF Subject
pdfkeywords={}, % PDF Keywords
pdfcreator={pdfLaTeX}, % PDF Creator
pdfproducer={LaTeX with hyperref and ClassicThesis} % PDF producer
}
 % Include the structure.tex file which specified the document structure and layout

\hyphenation{Fortran hy-phen-ation} % Specify custom hyphenation points in words with dashes where you would like hyphenation to occur, or alternatively, don't put any dashes in a word to stop hyphenation altogether

%----------------------------------------------------------------------------------------
%	TITLE AND AUTHOR(S)
%----------------------------------------------------------------------------------------

\title{\normalfont\spacedallcaps{Language reclamation: Literature review}} % The article title

%\subtitle{Subtitle} % Uncomment to display a subtitle

\author{\spacedlowsmallcaps{Caleb Moses*}} % The article author(s) - author affiliations need to be specified in the AUTHOR AFFILIATIONS block

\date{} % An optional date to appear under the author(s)

%----------------------------------------------------------------------------------------

\begin{document}

\title{MATH/COMP 562 ASSIGNMENT 4}
\date{Due: April 9th (Sunday)}
\author{Adam M. Oberman}
\maketitle

\section*{Generative Models}
Refer to \url{https://udlbook.github.io/udlbook/}

\subsection*{Exercise 4.1 (GANs)}

Given a dataset of genuine images, $S_m = \{x_1, \ldots, x_m\}$. The goal is to train a generator, $x_i = g(z_i, \theta)$, to generate images $x_i \in S_m$ from noise $z_i$. This is achieved by also training a discriminator, $f(x, \phi)$, which classifies an image as genuine or generated, using the minimax training loss
\begin{equation*}
\max_\theta \min_\phi \mathcal{L}^S_\theta(f(x, \phi), y)
\end{equation*}
applied to the combined dataset, $S_\theta = S_N \cup S_G^\theta$,
where
\begin{itemize}
    \item $S_N = \{(x_1, 1), \ldots, (x_m, 1)\}$, the (fixed) genuine images, with label $y = 1$,
    \item $S_G^\theta = \{(x^\theta_1, 0), \ldots, (x^\theta_m, 0)\}$, a changing set of generated images $x^\theta_i = g(z_i, \theta)$, with label $y = 0$.
\end{itemize}

%% \subsubsection*{Explicit GAN loss}

%% From \citet{prince2023understanding} the GAN loss function is as follows:

%% %% This is the characterisation of the saddle points
%% \begin{equation*}
%% \hat{\theta} = \textrm{argmax}_\theta \left[ \min_\phi \left[ \sum_j - \log(1 - D(G(z_j, \theta), \phi))) - \sum_i \log(D(x_i, \phi))) \right] \right]
%% \end{equation*}

%% where $\theta$ represents the weights of the generator function $G$, and $\phi$ is the weights of the discriminator $D$, and $\sigma$ is the sigmoid logistic function.

\subsubsection*{Exercise 4.1 (a)}
Write down and explain the loss for the discriminator, if the generator is fixed.

\subsubsection*{Solution 4.1 (a)}

From \citet{prince2023understanding} 15.5 the discriminator loss function is given by:

\begin{equation*}
  L(\phi) = - \sum_j \log\left(1 - \textrm{sig}(f(g(z_j, \theta), \phi)\right) - \sum_i log\left(\textrm{sig}(f(x_i, \phi))\right)
\end{equation*}

where ``$\textrm{sig}$'' represents the sigmoid function.

For a fixed generator, we can imagine each $g(z_i, \theta) = \hat{x}_i$ where $\hat{x}_i$ is an image generated according to $g$ which approximates $x_i$. We can simplify further if we let $D(x, \phi) = \textrm{sig}(f(x, \phi))$, basically considering the sigmoid classification layer as part of the discriminator function.

This gives us the loss function below:

\begin{equation*}
  L(\phi) = - \sum_j \log(1 - D(\hat{x}_j, \phi)) - \sum_i log(D(x_i, \phi))
\end{equation*}

The result is a binary cross-entropy loss where the goal is to distinguish $x_i$ from $\hat{x}_i$. Real samples $x_i$ decrease the loss by $log(D(x_i, \phi))$ while generated samples increase the loss by $\log(1-D(\hat{x}_j, \phi))$.

\subsubsection*{Exercise 4.1 (b)}
Write down and explain the loss (or gain, since it is a maximization) for the generator, if the discriminator is fixed.

\subsubsection*{Solution 4.1 (b)}

From \citet{prince2023understanding} 15.5 the generator gain function is given by:

\begin{equation*}
  L(\theta) = - \sum_j \log\left(1 - \textrm{sig}(f(g(z_j, \theta), \phi)\right)
\end{equation*}

where, as before ``$\textrm{sig}$'' represents the sigmoid function.

For a fixed discriminator, we can let $D_\phi(x) = \textrm{sig}(f(x, \phi))$ (we are bringing the $\phi$ into the function, since now it can't vary). In this scenario the GAN problem requires us to find the optimal $\hat{\theta}$ satisfying below:

\begin{equation*}
  L(\theta) = - \sum_j \log(1 - D_{\phi}(g(z_j, \theta)))
\end{equation*}

The purpose of the generator gain function is to reward the generator for samples that are assigned high probability by the discriminator.

\subsubsection*{Exercise 4.1 (c)}
Characterize the saddle points of the loss. If the generator outputs the full set of genuine images, is this optimal? What if the generator outputs just a subset of the genuine images?

A saddle point, $(\theta, \phi)$, is a point where (i) if we fix $\phi$ and change $\theta$, we cannot further increase the loss, and (ii) if we fix $\theta$ and change $\phi$, we cannot further decrease the loss.

\subsubsection*{Solution 4.1 (c)}

The saddle points of the GAN loss function are solutions to the following minimax problem, as described in \citet{prince2023understanding} equation 15.4:

\begin{equation*}
\hat{\theta} = \argmax_\theta \left[ \min_\phi \left[ \sum_j - \log(1 - \sig(f(g(z_j, \theta), \phi))) - \sum_i \log(\sig(f(x_i, \phi))) \right] \right]
\end{equation*}

This represents the Nash equilibrium in the minimax game between the generator and the discriminator.

If the generator produces the full set of genuine images, the discriminator could learn to memorize the genuine images and achieve perfect discrimination. In this scenario the generator would not be optimal.

If however the generator only produces a subset of the genuine images, then it is sub-optimal. This phenomenon is referred to as as "mode collapse", where the generator learns to generate only a limited variety of samples.

\newpage

\subsection*{Exercise 4.2 (Diffusion generative models)}
Given the dataset $S_m = \{x_1 , \ldots , x_m \}$, the goal is to generate images from the dataset. The inputs are a finite set of numbers $\alpha_t , t \in T$ and a finite set of noise vectors $\epsilon_j$ for $j \in J$.

The outputs are a set of models, $g_t : \mathcal{X} \rightarrow \mathcal{X}$, $g_t (x) = g(x, t)$ for $t \in T$.

For each $t \in T$ , let $S_t = \{x_{t,i,j}\}, i = 1, \ldots , m$ for $j \in J$ be the dataset of t-noisy images, $x_{t,i,j} = \sqrt{\alpha_t} x_i + \sqrt{1 - \alpha_t} \epsilon_j$

The training loss for $g_t$ is given by
\begin{equation*}
\hat{L}_{S_t}(g_t) = \frac{1}{m} \sum_{i=1}^{m} \sum_{j \in J} ||g_t (x_{t,i,j} ) - \epsilon_j ||^2
\end{equation*}

Define the closest point map
\begin{equation*}
  g^{CP}(x) = g^{CP}(x, S^m) = \argmin_{x_i \in S^m} ||x - x_i||^2
\end{equation*}

\subsubsection*{Exercise 4.2 (a)}
Assume that, for each $x_{t,i,j} \in S_t$, $g^{CP}(x_{t,i,j}) = x_i$. Find the loss minimizer $g^*_t$ and show that the loss is zero.

\subsubsection*{Solution to Exercise 4.2 (a)}
In this argument we will show that under the given assumption, the noise vector $\epsilon_j$ can be retrieved from $x_{t,i,j}$, and this fact can be used to construct a loss minimizing function.

The training loss is the mean squared error between the output of $g_t(x_{t,i,j})$ and the noise vectors $\epsilon_j$:

\begin{equation*}
  \hat{L}_{S_t}(g_t) = \frac{1}{m} \sum_{i=1}^{m} \sum_{j \in J} ||g_t (x_{t,i,j} ) - \epsilon_j ||^2
\end{equation*}

Minimizing this loss amounts to minimizing $||g_t (x_{t,i,j})-\epsilon_j||^2$ for every $i = 1, ..., m$, and every $j \in J$.

Since $x_{t,i,j} = \sqrt{\alpha_t}x_i + \sqrt{1-\alpha_t}\epsilon_j$, rearranging for $\epsilon_j$ we can construct a new function $g^*$ which will depend only on $x_{t,i,j}$ and $t$:

\begin{align*}
  \epsilon_j &= \frac{x_{t,i,j} - \sqrt{\alpha_t}x_i}{\sqrt{1-\alpha_t}} \\
   &= \frac{x_{t,i,j} - \sqrt{\alpha_t} \cdot \argmin_{x_k \in S^m} ||x_{t,i,j}-x_k||^2}{\sqrt{1-\alpha_t}} \\
  &= g^*_t(x_{t,i,j})
\end{align*}

Where we take advantage of the fact that $g^{CP}(x_{t,i,j}) = x_i$ to remove the dependency on $x_i$. By construction $g$ depends only on $x_{t,i,j}$ and $t$. This means $g^*_t(x)$ is a viable candidate minimizer for the loss $\hat{L}_{S_t}$.

Lastly we show that $g^*_t$ minimizes $\hat{L}_{S_t}$:

\begin{align*}
  \hat{L}_{S_t}(g^*_t) &= \frac{1}{m} \sum_{i=1}^{m} \sum_{j \in J} ||g^*_t (x_{t,i,j} ) - \epsilon_j ||^2 \\
  &= \frac{1}{m} \sum_{i=1}^{m} \sum_{j \in J} ||\epsilon_j - \epsilon_j ||^2 \\
  &= \frac{1}{m} \sum_{i=1}^{m} \sum_{j \in J} 0 = 0
\end{align*}

Since the loss is non-negative, we can conclude that $g^*_t$ minimizes the loss.

\subsubsection*{Exercise 4.2 (b)}
In the case above, are there any other loss minimizers?

\subsubsection*{Solution to Exercise 4.2 (b)}

Suppose there are two different minimizers $g^t$ and $h^t$ of the loss. They both achieve the minimum loss, which is zero. So for all $i = 1, ..., m$ and for all $j \in J$,

\begin{equation*}
  ||g^t(x_{t,i,j}) - \epsilon_j||^2 = ||h^t(x_{t,i,j}) - \epsilon_j||^2 = 0
\end{equation*}

This implies that $g^t(x_{t,i,j}) = h^t(x_{t,i,j}) = \epsilon_j$ for all $i$ and $j$. Therefore, if $g^t$ and $h^t$ are both loss minimizers for $\hat{L}_{S^t}$, they must be equal. Therefore $g^*_t$ is unique.

\subsubsection*{Exercise 4.2 (c)}
How do the loss minimizers for this generative model compare to the loss minimizers for GANs? Which is better in terms of coverage of the dataset?

\subsubsection*{Solution to Exercise 4.2 (c)}

In GANs, the generator aims to fool the discriminator by producing outputs that are indistinguishable from the real data, thus the loss minimizer corresponds to a generator that perfectly replicates the real data distribution. However, GANs can suffer from mode collapse, where they only generate a small subset of the real data, failing to cover the entire data distribution.

In contrast, diffusion generative add noise to the real data and learns to reverse this process, mapping noised images back to the originals. In this case, the loss minimizer is a deterministic function that directly maps the noisy data back to the closest point in the original dataset. This approach virtually guarantees coverage over the original data, as long as the original images are not too close together.

In terms of dataset coverage, diffusion generative models generally perform better. They more directly address mode collapse, which is a notable issue with GANs.

\newpage

\subsection*{Exercise 4.3 (Normalizing Flows: transformation of densities)}
Problems 16.1 and 16.2 from ”Understanding Deep Learning”

\subsection*{Exercise 4.4 (Normalizing Flows: Fixed point theorem)}
Problem 16.11 from Chapter 16 of ”Understanding Deep Learning”

\subsection*{Exercise 4.5 (Diffusion generative models)}
Problems 18.1 and 18.2 from Chapter 18 of ”Understanding Deep Learning”

%----------------------------------------------------------------------------------------
%	BIBLIOGRAPHY
%----------------------------------------------------------------------------------------

\bibliography{sample.bib} % The file containing the bibliography


\end{document}
